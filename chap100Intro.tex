
\chapter{Introdução}\label{key:introducao}

\textbf{\textcolor{Red}{Documento em construção... \\O texto neste capítulo explica a estrutura do documento. Depois de leres e compreenderes deves alterar o texto contextualizado com o tema do teu projeto. No final deves remover este comentário.}}


Se necessitares de ajuda para escreveres em \LaTeX\ podes e deves consultar o capítulo com o código exemplo na página \pageref{key:latexsample}.

[A introdução deve, tal como o próprio nome indica, introduzir o tema do trabalho. Não deve haver pressa em falar da empresa onde foi realizado o estágio ou o curso a que se refere o trabalho. Deve fazer-se uma introdução à área, Os Sistemas Informáticos ou as Ciências da Computação são áreas bastante grandes, pelo que não se deve supor que o leitor está a par das necessidades ou das tecnologias usadas em determinada área. No entanto, não devem ser explicados conceitos básicos, que qualquer licenciado numa engenharia de sistemas informáticos ou em ciências da computação tenham obrigação de conhecer.

Na formatação do texto tente-se que não existam demasiadas zonas em branco. Não é pelo número de páginas que se mede a qualidade de um relatório. E, uma vez que os documentos são impressos, poupar algumas folhas é económico e ecológico. 

Relembra-se que todo o conteúdo do documento deve ser original. Quaisquer citações retiradas de algum livro ou sítio da Internet devem ser devidamente formatadas, e a referência bibliográfica adicionada \citep{knuth1973}:

\emph{By understanding a machine-oriented language, the programmer will tend to use a much more efficient method; it is much closer to reality. }

Do mesmo modo, se algum texto, embora usando palavras do autor do documento, refira alguma ideia defendida por um outro autor, num outro documento, então também deverá aparecer a respetiva referência bibliográfica (PennState University Libraries, 2017). 

O uso de citações é especialmente útil para defender ideias que outros autores também defendem, e que o autor do documento não tem com provar.] 

\section{Objetivos}
[Numa pequena secção da introdução liste, cuidadosamente, os objetivos do trabalho. Não confundir com os requisitos do software. Apenas o que se pretendia atingir originalmente.] 
\section{Contexto}
[No caso de um estágio, é nesta secção que se deverá falar da empresa em que o estágio foi realizado. Se o projeto desenvolvido faz parte de um projeto mais amplo, faz sentido que se documente os objetivos do projeto com um todo, de modo que o leitor consiga perceber onde o trabalho realizado encaixa.] 
\section{Estrutura do documento}
[A última secção da introdução deve explicar a estrutura do documento: quais são só capítulos existentes (para além do primeiro) e o que será discutido em cada um desses capítulos. A estrutura típica de um relatório de desenvolvimento de software é: 

Introdução, com um breve resumo do que se pretende atingir, e uma descrição clara dos objetivos;

\begin{enumerate}
    \item Análise ao problema, que poderá incluir uma análise ao estado da arte ou ao modelo de negócio onde se pretende intervir;
    \item Análise e modelação do sistema, em que sejam levantados sistematicamente os requisitos, descritos diagramas de caso de uso e de atividade (que descrevam/formalizem o modelo de negócio). 
    \item Implementação, em que se descrevam as tecnologias escolhidas (e se justifiquem), e se refira detalhes sobre a implementação.
    \item Análise de resultados e testes, seja uma análise/avaliação aos resultados obtidos, sejam testes de usabilidade ou unitários ao trabalho desenvolvido. 
    \item Conclusão.]
\end{enumerate}{}
