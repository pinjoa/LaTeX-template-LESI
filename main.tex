% ---
% template desenvolvido inicialmente por (DC-LESI) Prof Alberto Simões
% modificado/melhorado por João Carlos Pinto (#20808) LESI-PL 2020-23
% ---
\documentclass[a4paper,12pt,twoside]{book}
\usepackage{lesi/lesi}
\usepackage{fancyvrb}

\title{Projeto Unicórnio}
% utilizar \AND para mais do que um autor
\author{João Carlos Pinto, 20808}
\LESI
\regimePosLaboral         % ou \regimeDiurno
\date{\today}

%% Caso tenham mais que um orientador, colocar \AND
\orientador{Alberto Simões}

%% comentar estas três linhas para projectos
% \empresa{Fuzzy Bit - Software Engineering}
% \enderecoEmpresa{Barcelos, Portugal}
% \supervisor{Eng. A. U. Thor}

% comentar se não for para usar glossários
\makeglossaries

% inicio do documento
\begin{document}
\input{glossario}

 

\newacronym{avac}{AVAC}{Aquecimento, ventilação e ar condicionado}
\newacronym{cctv}{CCTV}{Closed-circuit television — Circuito fechado de televisão}
\newacronym{cd}{CD}{Continuous Delivery — Entrega contínua}
\newacronym{ci}{CI}{Continuous Integration — Integração contínua}
\newacronym{cis}{CIS}{Center for Internet Security — Centro de segurança para a internet}
\newacronym{ciso}{CISO}{Chief Information Security Officer — Responsável de Segurança de Informação}
\newacronym{cobit}{COBIT}{Control Objectives for Information and Related Technologies — Objetivos de controlo para informações e tecnologias relacionadas}
\newacronym{coo}{COO}{Chief Operations Officer — Responsável das Operações}
\newacronym{crc}{CRC}{Cyclic Redundancy Check — Verificação cíclica de redundância}
\newacronym{csc}{CSC}{Critical Security Controls — Controlos críticos de segurança}
\newacronym{csirt}{CSIRT}{Computer Security Incident Response Team — Equipa de Resposta a Incidentes de Segurança Informática}
\newacronym{cve}{CVE}{Lista de registos que contêm um número de identificação, uma descrição e, pelo menos, uma referência pública para vulnerabilidades de segurança}
\newacronym{dlp}{DLP}{Data Loss Prevention — Prevenção de perda de informação}
\newacronym{dmz}{DMZ}{Demilitarized Zone — Zona desmilitarizada}
\newacronym{dns}{DNS}{Domain Name System — Sistema de resolução de nomes de domínio}
\newacronym{ensc}{ENSC}{Estratégia Nacional de Segurança do Ciberespaço 2019-2023}
\newacronym{ftp}{FTP}{File Transfer Protocol (Protocolo de Transferência de Ficheiros) }
\newacronym{http}{HTTP}{HyperText Transfer Protocol (Protocolo de Transferência de Hipertexto)}
\newacronym{ids}{IDS}{Intrusion Detection System — Sistema de deteção de intrusões}
\newacronym{iot}{IoT}{Internet of Things — Internet das coisas}
\newacronym{ip}{IP}{Internet Protocol — Protocolo de comunicações}
\newacronym{ips}{IPS}{Intrusion Prevention System — Sistema de prevenção de intrusões}
\newacronym{isaca}{ISACA}{Information Systems Audit and Control Association — Associação de auditoria e controlo de sistemas de informação}
\newacronym{iso}{ISO}{International Organization for Standardization — Organização internacional de normalização}
\newacronym{iso/iec}{ISO/IEC}{International Organization for Standardization/lnternational Electrotechnical Commission — Organização internacional de normalização/Comissão eletrotécnica internacional}
\newacronym{mitre}{MITRE}{Base de dados de vulnerabilidades mantida por organização não governamental Norte-americana, internacionalmente reconhecida como a líder nesta matéria}
\newacronym{nist}{NIST}{National Institute of Standards and Technology — Instituto Nacional de Padrões e Tecnologia (Norte-americano)}
\newacronym{qnrcs}{QNRCS}{Quadro Nacional de Referência para a Cibersegurança}
\newacronym{raci}{RACI}{Responsible — Responsável, Accountable — Aprovador, Consulted — Consultado e Informed— Informado. Matriz de atribuição de Responsabilidades}
\newacronym{soc}{SOC}{Security Operations Center— Centro de Operações de Segurança}
\newacronym{sri}{SRI}{Segurança das Redes e da Informação}
\newacronym{swot}{SWOT}{Strengths — Forças, Weaknesses — Fraquezas, Opportunities — Oportunidades, Threats — Ameaças}
\newacronym{vpn}{VPN}{Virtual Private Network — Rede privada virtual}
\newacronym{sgsi}{SGSI}{Sistema de Gestão de Segurança da Informação}
\newacronym{tcp}{TCP}{Transmission Control Protocol}
\newacronym{ti}{TI}{Tecnologias de Informação}
\newacronym{ups}{UPS}{Uninterruptible Power Source — Unidade de alimentação ininterrupta}
\newacronym{waf}{WAF}{Web Application Firewall — Firewall de aplicações web}
\newacronym{www}{WWW}{World Wide Web — Rede mundial de computadores}
 

\frontmatter
\maketitle  % print the title

\begin{resumo}

\textbf{\textcolor{Red}{Documento em construção... \\Falta completar esta secção! \\No final deves remover este comentário.}}

Resumo do trabalho realizado. Deve ser sucinto, e cobrir todo o relatório: uma introdução ao problema que se pretendeu resolver, um pequeno resumo da abordagem realizada, e algumas conclusões do trabalho atingido.

Poderão ser criados vários parágrafos, até para que cada um corresponda às três fases de introdução, desenvolvimento e conclusão.

Não é relevante colocar no resumo o local de estágio ou a referência ao curso. Essa informação já consta da capa.
\end{resumo}

%\begin{abstract}
%This is the translation of the previous text. It should say the exact same thing. Please do not use directly Google Translator.
%\end{abstract}

%% Comment the following part if you are not acknowledging anybody
%\begin{agradecimentos}
%[A secção de agradecimentos é a parte pessoal do documento, e o único sítio onde o aluno pode escrever de forma menos formal, usando o tipo de linguagem que lhe parecer adequado para as pessoas a quem agradece.] 
%\end{agradecimentos}

\tableofcontents

% comentar se nao tiver figuras
\listoffigures
\addcontentsline{toc}{chapter}{Lista de Figuras}

% comentar se nao tiver tabelas
\listoftables
\addcontentsline{toc}{chapter}{Lista de Tabelas}

% comentar se nao se quiser lista de listagens
\lstlistoflistings
\addcontentsline{toc}{chapter}{Lista de Código}

% Commentar proximas duas linhas se nao for para usar acronimos
\printglossary[type=\acronymtype,title={Siglas \& Acrónimos},toctitle={Siglas \& Acrónimos}]

% Commentar proximas duas linhas se nao for para usar glossarios
\printglossary[title={Indice de Termos},toctitle={Indice de Termos}]

\mainmatter

\input{chap100Intro}

\chapter{Segundo capitulo}

Blá, blá, blá...

\textbf{\textcolor{Red}{Documento em construção... \\Este capítulo está vazio e é intencional, deves remover ou substituir o conteúdo e apagar este comentário.}}

%-----------------------------
\chapter{\label{key:latexsample}Diferentes exemplos num único capítulo}

Esta é a segunda versão do documento utilizado como template para o relatório de trabalhos e/ou projetos desenvolvidos para cada uma das Unidades Curriculares da Licenciatura de Engenharia de Sistemas Informáticos na Escola Superior de Tecnologia\footnote{\url{https://est.ipca.pt/}}.


\noindent\fbox{%
	\parbox{\textwidth}{%
		\textcolor{red}{NOTA: Documento em construção! Este capítulo~\ref{key:latexsample} deve ser removido na versão final deste documento...}
	}%
}


%-----------------------------
\section{Preparar o ambiente LaTeX}

Assumindo que estás a utilizar o sistema operativo Windows, para editares documentos \LaTeX\ necessitas de instalar as seguintes ferramentas:
\begin{itemize}
	\item \textbf{MiKTeX}\footnote{\url{https://miktex.org/}}, esta ferramenta funciona na linha de comandos (de um terminal) e disponibiliza um conjunto de comandos necessários para compilar o código e produzir o documento final. No repositório tens dois scripts\footnote{no formato \textit{batch processing file}, mas também foram criados para utilizar num terminal do MacOS} para gestão do projeto:
	\begin{itemize}
		\item \textbf{tex-win-make.bat} executa a sequência completa para compilar o projeto;
		\item \textbf{tex-win-clear-temp-files.bat} limpa os ficheiros desnecessários criados durante a compilação do projeto.
	\end{itemize}
	\item \textbf{TeXstudio}\footnote{\url{https://texstudio.org/}}, é um excelente IDE para \LaTeX.
\end{itemize}


%-----------------------------
\section{Utilizar referências}

Este capítulo fala de \gls{stemmer}s. Mas não esquecer os \Gls{lematizador}es

O \acrfull{http} é um protocolo baseado em \acrshort{tcp}.

Citar o livro "The Art of Computer Programming"\cite{knuth1973}.

A introdução está colocada na página \pageref{key:introducao}.


%-----------------------------
\section{Secção}

Blá, blá, blá...

\colorbox{Red}{documento em construção...}

\subsection{Subsecção}

Blá, blá, blá...

\textbf{\textcolor{Red}{documento em construção...}}


\subsubsection{Subsubsecção}

Blá, blá, blá...


%-----------------------------
\section{Utilizar lista1}
Texto de suporte para um exemplo de utilização de lista.

\begin{itemize}
	\item Item 1;
	\item Item 2; 
	\item Item 3;
	\item Conclusão.
\end{itemize}{}


%-----------------------------
\section{Utilizar lista2}
Texto de suporte para outro exemplo de utilização de lista.

\begin{enumerate}
	\item Item 1;
	\item Item 2; 
	\item Item 3;
	\item Conclusão.
\end{enumerate}{}

%-----------------------------
\section{Exemplo de colocação de figuras}

Ao contrário do Word, o \LaTeX{} usa um mecanismo de colocação de figuras e tabelas em que estas flutuam ao longo das páginas de acordo com a necessidade/disponibilidade em termo de espaço vertical.
Assim, não devem usar frases como ``na figura acima'', ou ``na figura abaixo'', mas fazer referências:
``tal como se pode observar na Figura~\ref{fig:1}'' (a figura poderá estar numa página diferente, portanto se for muito importante indicar a página, necessitas apenas de colocar a referência para essa página \pageref{fig:1} caso seja pertinente e necessário).

\begin{figure}[htb]
	\centering
	\includegraphics[width=0.8\linewidth]{img/sample}  % largura percentual 
	\caption{Figura para o exemplo}
	\label{fig:1}
\end{figure}


%-----------------------------
\section{Exemplo de tabela}
Texto de suporte para um exemplo de tabela.

O mesmo acontece com as tabelas, como se pode ver na Tabela~\ref{tab:1} (a tabela poderá estar numa página diferente, portanto se for muito importante indicar a página, necessitas apenas de colocar a referência para essa página \pageref{tab:1} caso seja pertinente e necessário).

\begin{table}[htb]
	\centering
	\begin{tabular}{ccccc}
		\toprule
		\textbf{A} & \textbf{B} & \textbf{C} & \textbf{D} & \textbf{Total} \\
		\midrule
		1 & 2 & 3 & 4 & 10  \\
		2 & 3 & 4 & 5 & 14  \\
		3 & 4 & 5 & 6 & 18  \\
		4 & 5 & 6 & 7 & 22  \\
		\bottomrule
	\end{tabular}
	\caption{Tabela exemplo}
	\label{tab:1}
\end{table}


%-----------------------------
\section{Utilizar Verbatim}
Texto de suporte para o exemplo de utilização do bloco Verbatim.

\subsection{Lista dos carateres a ignorar}
\begin{Verbatim}
  t_ignore = " \t\n"
\end{Verbatim}
\hfill \break

%-----------------------------
\section{Utilizar listagem de programa}
Texto de suporte para o exemplo de utilização de listagem.

\subsection{Listagem de programa em C\#}
Para a inclusão de código, usa-se algo semelhante. Veja-se a Listagem~\ref{lst:1}.

\begin{lstlisting}[language={[sharp]c},
caption={Método para contar o número de elementos numa lista iguais a uma determinada string.},
label=lst:1]
  public int count(string x) {
    return items.Select( y => y == x ).Count();
  }
\end{lstlisting}

\subsection{Listagem de programa em C}
Outro exemplo de código, usa-se algo semelhante. Veja-se a Listagem~\ref{lst:p1e}.

\begin{lstlisting}[language={c},
	caption={Código do programa: \textbf{apaga}.},
label=lst:p1e]
/**
* programa apaga.c
* @author #20808 Joao Carlos Pinto
*/
#include <stdio.h>
#include <unistd.h>
#include <errno.h>
#include <string.h>
#include "mytools.h"

int main(int argc, char *argv[]) {
  char buffer[MAXBUFFERSIZE];
  if (argc < 2) {
    buffer[0] = '\0';
    sprintf(buffer, "Falta: ficheiro\n Deve "+
    "utilizar-se desta forma:\n%s ficheiro(s)\n", 
    argv[0]);
    escrevErro(buffer);
    return 1;
  }
  int resultado, bkErrno;
  for(int i=1; i<argc; i++){
    if ((resultado = unlink(argv[i])) == -1) {
      buffer[0] = '\0';
      bkErrno = errno;
      sprintf(buffer, "erro (%d, %s) ao apagar o "+
      "ficheiro: %s\n", bkErrno, strerror(bkErrno), 
      argv[i]);
      escrevErro(buffer);
      return 1;
    }
    if (resultado == 0) {
      buffer[0] = '\0';
      sprintf(buffer, "o ficheiro \"%s\" foi apagado!\n", 
      argv[i]);
      escrever(buffer);
    } else {
      buffer[0] = '\0';
      bkErrno = errno;
      sprintf(buffer, "resultado (%d, %s) inesperado ao "+
      "apagar o ficheiro \"%s\"!\n", bkErrno, 
      strerror(bkErrno), argv[i]);
      escrevErro(buffer);
      return 1;
    }
  }
  return 0;
}
\end{lstlisting}



% comentar se não for para usar bibliografia
\bibliography{biblio}
\addcontentsline{toc}{chapter}{Bibliografia}

\end{document}  
